\documentclass{article}
\usepackage[utf8]{inputenc}
\usepackage[spanish]{babel}
\usepackage{listings}
\usepackage{graphicx}
\graphicspath{ {images/} }
\usepackage{cite}

\begin{document}

\begin{titlepage}
    \begin{center}
        \vspace*{1cm}
            
        \Huge
        \textbf{Calistenia}
            
        \vspace{0.5cm}
        \LARGE
        Parcial 1
            
        \vspace{1.5cm}
            
        \textbf{Juan José Florez Argaez}
        \textbf{C.C: 1001765286}
        
            
        \vfill
            
        \vspace{0.8cm}
            
        \Large
        Despartamento de Ingeniería Electrónica y Telecomunicaciones\\
        Universidad de Antioquia\\
        Medellín\\
        Marzo de 2021
            
    \end{center}
\end{titlepage}

\tableofcontents
\newpage
\section{Introducción}\label{intro}
Este es un interesante desafío con el cual podemos demostrar la aplicabilidad de un algoritmo a la vida real.

\section{Objetivo}\label{intro}
Nuestro desafió consiste en llevar dos objetos de iguales características (Tarjetas) que se encuentran en un estado 'A' a un estado 'B' a través de una serie de pasos guiados.
\subsection{Condiciones}
Asumiendo que tenemos dos tarjetas de iguales características apoyadas encima de una mesa y con una hoja de papel tapando las tarjetas 'Estado A' debemos con una sola mano seguir las instrucciones para lograr completar el desafío --> Armar una pirámide cuadrangular 'estado B'.

\section{Pasos a seguir} \label{contenido}
A continuación siga los pasos para completar de manera exitosa el desafió. 

\begin{enumerate}
    \item Mover la hoja de papel a un lado de las tarjetas.
    \item Tomar las dos tarjetas de tal manera que el dedo pulgar (dedo gordo) de la mano con la que esta realizando el desafío sujete un extremo lateral de las tarjetas, los dedos medio, anular y meñique sostengan el extremo lateral contrario y el dedo índice (dedo con el que apunta) toque el lateral que hay en medio.
    \item Con las tarjetas agarradas como se indico en el paso anterior hacemos presión con el dedo índice a las tarjetas en medio de la hoja mencionada en el paso 1 contra la mesa. 
    \item Soltamos el dedo pulgar y los dedos medio, anular y meñique asegurándonos que el el dedo índice sostenga las tarjetas.
    \item Con los dedos que nos quedaron libres tras realizar el paso 4 (dedo pulgar y los dedos medio, anular y meñique) agarramos solo una de las tarjetas.
    \item Separamos las tarjetas hasta realizar la figura de una pirámide cuadrangular, intentarlo hasta lograrlo.
\end{enumerate}



\bibliographystyle{IEEEtran}


\end{document}
